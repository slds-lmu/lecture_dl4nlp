\input{../../style/preamble}
\input{../../latex-math/basic-math.tex}
\input{../../latex-math/basic-ml.tex}

\newcommand{\titlefigure}{figure/73-gpt3.jpg}

\newcommand{\learninggoals}{
\item Overview on relevant tasks and benchmarks
\item GPT-3's performance in X-shot settings}

\definecolor{texblue}{rgb}{0, 0, 1}
\def\myblue#1{\textcolor{texblue}{#1}}

\title{Generative Pre-Trained Transformers}
% \author{}
\institute{\href{https://slds-lmu.github.io/lecture_dl4nlp/}{slds-lmu.github.io/lecture\_dl4nlp}}
\date{}

\begin{document}
\lecturechapter{GPT-3: Tasks \& Performance}
\lecture{Deep Learning for NLP}

% ------------------------------------------------------------------------------

\begin{vbframe}{Lambada task}

\vfill

	\begin{figure}
		\centering
		\includegraphics[width=10cm]{figure/lambadaformat.png}\\
		\citebutton{Source: Brown et al., 2020}{https://arxiv.org/abs/2005.14165}
	\end{figure}

\vfill

\end{vbframe}

% ------------------------------------------------------------------------------

\begin{vbframe}{Performance on lambada}

\vfill

	\begin{figure}
		\centering
		\includegraphics[width=10cm]{figure/lambadaperf.png}\\
		\citebutton{Source: Brown et al., 2020}{https://arxiv.org/abs/2005.14165}
	\end{figure}

\vfill

\end{vbframe}

% ------------------------------------------------------------------------------

\begin{vbframe}{``Closed book'' question answering (QA)}

\vfill
			
	\begin{figure}
		\centering
		\includegraphics[width=10cm]{figure/triviaformat.png}\\
		\citebutton{Source: Brown et al., 2020}{https://arxiv.org/abs/2005.14165}
	\end{figure}
	
\vfill

\end{vbframe}

% ------------------------------------------------------------------------------

\begin{vbframe}{Performance on the closed-book QA}

\vfill

	\begin{figure}
		\centering
		\includegraphics[width=10cm]{figure/triviaperf.png}\\
		\citebutton{Source: Brown et al., 2020}{https://arxiv.org/abs/2005.14165}
	\end{figure}

\vfill

\end{vbframe}

% ------------------------------------------------------------------------------

\begin{vbframe}{Performance on machine translation}

\vfill

	\begin{figure}
		\centering
		\includegraphics[width=10cm]{figure/mtperf.png}\\
		\citebutton{Source: Brown et al., 2020}{https://arxiv.org/abs/2005.14165}
	\end{figure}

\vfill

\end{vbframe}

% ------------------------------------------------------------------------------

\begin{vbframe}{Winograd task}

\vfill

	\begin{figure}
		\centering
		\includegraphics[width=10cm]{figure/winogradformat.png}\\
		\citebutton{Source: Brown et al., 2020}{https://arxiv.org/abs/2005.14165}
	\end{figure}

\vfill

\end{vbframe}


% ------------------------------------------------------------------------------

\begin{vbframe}{Performance on Winograd task}

\vfill

	\begin{figure}
		\centering
		\includegraphics[width=10cm]{figure/winogradperf.png}\\
		\citebutton{Source: Brown et al., 2020}{https://arxiv.org/abs/2005.14165}
	\end{figure}

\vfill

\end{vbframe}


% ------------------------------------------------------------------------------

\begin{vbframe}{ARC task}

\vfill

\begin{itemize}
	\item Multipe-Choice questions from 3rd to 9th grad scince exams
	\item \textit{Challenge} version: Filtered to questions which simple statistical or information retrieval methods are unable to correctly answer
\end{itemize}

	\begin{figure}
		\centering
		\includegraphics[width=10cm]{figure/arcformat.png}\\
		\citebutton{Source: Brown et al., 2020}{https://arxiv.org/abs/2005.14165}
	\end{figure}

\vfill

\end{vbframe}


% ------------------------------------------------------------------------------

\begin{vbframe}{Performance on ARC task}

\vfill

	\begin{figure}
		\centering
		\includegraphics[width=10cm]{figure/arcperf.png}
	\end{figure}

\vfill

\end{vbframe}


% ------------------------------------------------------------------------------

\begin{vbframe}{RACE task}

\vfill

	\begin{figure}
		\centering
		\includegraphics[width=8cm]{figure/raceformat,h.png}
	\end{figure}
	
\vfill

\end{vbframe}



% ------------------------------------------------------------------------------

\begin{vbframe}{Performance on RACE task}

\vfill

	\begin{figure}
		\centering
		\includegraphics[width=10cm]{figure/raceperf.png}
	\end{figure}

\vfill

\end{vbframe}



% ------------------------------------------------------------------------------

\begin{vbframe}{Performance on SuperGLUE task}

\vfill

	\begin{figure}
		\centering
		\includegraphics[width=10cm]{figure/superglueperf.png}
	\end{figure}

\vfill

\end{vbframe}



% ------------------------------------------------------------------------------

\begin{vbframe}{SuperGLUE}

\vfill

  \begin{itemize}
\item BoolQ
\item CB (true/false/neither)
\item COPA
\item RTE (similar to natural language inference)
\item WiC 
\item WSC
\item MultiRC (true/false)
\item ReCoRD 
    \end{itemize}

\vfill

\end{vbframe}



% ------------------------------------------------------------------------------

\begin{vbframe}{BoolQ (Boolean Question) task}

\vfill

	\begin{figure}
		\centering
		\includegraphics[width=10cm]{figure/boolqtask.png}
	\end{figure}

\vfill

\end{vbframe}



% ------------------------------------------------------------------------------

\begin{vbframe}{WiC (Word in Context) task}

\vfill

	\begin{figure}
		\centering
		\includegraphics[width=10cm]{figure/wicperf.png}
	\end{figure}

\vfill

\end{vbframe}



% ------------------------------------------------------------------------------

\begin{vbframe}{COPA task}

\vfill

	\begin{figure}
		\centering
		\includegraphics[width=10cm]{figure/copaperf.png}
	\end{figure}

\vfill

\end{vbframe}



% ------------------------------------------------------------------------------

\begin{vbframe}{WSC (Winograd Schema Challenge) task}

\vfill

	\begin{figure}
		\centering
		\includegraphics[width=10cm]{figure/wscformat.png}
	\end{figure}

\vfill

\end{vbframe}



% ------------------------------------------------------------------------------

\begin{vbframe}{ReCoRD task}

\vfill

	\begin{figure}
		\centering
		\includegraphics[width=10cm]{figure/recordformat.png}
	\end{figure}

\vfill

\end{vbframe}



% ------------------------------------------------------------------------------

\begin{vbframe}{ANLI task}

\vfill

	\begin{figure}
		\centering
		\includegraphics[width=10cm]{figure/anliformat.png}
	\end{figure}

\vfill

\end{vbframe}



% ------------------------------------------------------------------------------

\begin{vbframe}{Performance on ANLI task}

\vfill

	\begin{figure}
		\centering
		\includegraphics[width=10cm]{figure/anliperf.png}
	\end{figure}

\vfill

\end{vbframe}



% ------------------------------------------------------------------------------

\begin{vbframe}{SAT Analogies task}

\vfill

	\begin{figure}
		\centering
		\includegraphics[width=10cm]{figure/satformat.png}
	\end{figure}

\vfill

\end{vbframe}



% ------------------------------------------------------------------------------

\begin{vbframe}{Performance on SAT Analogies}

\vfill

	\begin{figure}
		\centering
		\includegraphics[width=10cm]{figure/satperf.png}
	\end{figure}

\vfill

\end{vbframe}



% ------------------------------------------------------------------------------

\begin{vbframe}{GPT3 can correct grammar}

\vfill

	\begin{figure}
		\centering
		\includegraphics[width=10cm]{figure/correctgrammar.png}
	\end{figure}

\vfill

\end{vbframe}



% ------------------------------------------------------------------------------

\begin{vbframe}{Generation of news articles}

\vfill

  \begin{itemize}
\pause\item Context given to gpt3:
  \begin{itemize}
\pause\item Three ``training'' articles to condition gpt3
\pause\item Title and subtitle of a 4th article
    \end{itemize}
\pause\item gpt3 then has to generated the body of the 4th article
\pause\item Evaluation: Humans are presented the
human-generated original article and the gpt3-generated
article and are asked to identify which is fake.
    \end{itemize}

\vfill

\end{vbframe}



% ------------------------------------------------------------------------------

\begin{vbframe}{Humans cannot distinguish human generated vs GPT-3 generated}

\vfill

	\begin{figure}
		\centering
		\includegraphics[width=10cm]{figure/humanvsmachine.png}
	\end{figure}

\vfill

\end{vbframe}



% ------------------------------------------------------------------------------

\begin{vbframe}{Hard to identify as fake}

\vfill

	\begin{figure}
		\centering
		\includegraphics[width=10cm]{figure/hardtoidentify.png}
	\end{figure}

\vfill

\end{vbframe}



% ------------------------------------------------------------------------------

\begin{vbframe}{Easier to identify as fake}

\vfill

	\begin{figure}
		\centering
		\includegraphics[width=10cm]{figure/easytoidentify.png}
	\end{figure}

\vfill

\end{vbframe}



% ------------------------------------------------------------------------------

\begin{vbframe}{Summary}

\vfill

  \begin{itemize}
\item The average person has difficulty distinguishing
human-generated and gpt3-generated news.
\item However, the non-average person probably can
distinguish them quite well.
\item There's also evidence that machines
are able to distinguish
human-generated and gpt3-generated news.
\item This has great significance for preventing abuse
of AI technology.
  \end{itemize}

\vfill

\end{vbframe}


\endlecture
\end{document}
