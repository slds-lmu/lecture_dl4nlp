\input{../../style/preamble}
\input{../../latex-math/basic-math.tex}
\input{../../latex-math/basic-ml.tex}

\newcommand{\titlefigure}{figure/sesamestreet.jpeg}
\newcommand{\learninggoals}{
\item Understand the developments of the post-BERT era
\item Get to know different self-supervised objectives
\item Understand how to tackle BERTs critical shortcomings}

\title{Post-BERT Era}
% \author{}
\institute{\href{https://slds-lmu.github.io/lecture_dl4nlp/}{slds-lmu.github.io/lecture\_dl4nlp}}
\date{}

\begin{document}
\lecturechapter{BERT-based architectures}
\lecture{Deep Learning for NLP}

% ------------------------------------------------------------------------------

\begin{frame}{Sucessors of BERT}
\hbox{\hspace{-0.5em} \includegraphics[width=12cm,page=1]{figure/transfer_learning_timeline3_nlp.pdf}}
\end{frame}
\begin{frame}[noframenumbering]{Predecessors of BERT}
\hbox{\hspace{-0.5em} \includegraphics[width=12cm,page=2]{figure/transfer_learning_timeline3_nlp.pdf}}
\end{frame}
\begin{frame}[noframenumbering]{Predecessors of BERT}
\hbox{\hspace{-0.5em} \includegraphics[width=12cm,page=3]{figure/transfer_learning_timeline3_nlp.pdf}}
\end{frame}
\begin{frame}[noframenumbering]{Predecessors of BERT}
\hbox{\hspace{-0.5em} \includegraphics[width=12cm,page=4]{figure/transfer_learning_timeline3_nlp.pdf}}
\end{frame}

% ------------------------------------------------------------------------------

\begin{frame}{roberta -- pre-training improvements}

\vfill

\textbf{R}obustly \textbf{o}ptimizied \textbf{BERT} \textbf{a}pproach \citebutton{Liu et al., 2019}{https://arxiv.org/abs/1907.11692}\\ \medskip\medskip

\textbf{Short summary:}

	\begin{itemize}
		\item Change of the \texttt{MASK}ing strategy  \\
					$\rightarrow$ BERT masks the sequences once before pre-training  \\
					$\rightarrow$ RoBERTa uses dynamic \texttt{MASK}ing  \\
					$\Rightarrow$ RoBERTa sees the same sequence \texttt{MASK}ed differently
		\item RoBERTa does not use the additional NSP objective during pre-training
		\item Authors claim that BERT is seriously "undertrained"
			\begin{itemize}
				\item 160 GB pre-training corpus instead of 13 GB
				\item Pre-training is performed with larger batch sizes (8k)
			\end{itemize}
	\end{itemize}
	
\vfill

\end{frame}

% ------------------------------------------------------------------------------

\begin{frame}{Dynamic vs. Static Masking}

\vfill

	\textbf{Static Masking (BERT):}

	\begin{itemize}
		\item Apply \texttt{MASK}ing procedure to pre-training corpus once
		\item (additional for BERT: Modify the corpus for NSP)
		\item Train for approximately 40 epochs
	\end{itemize}

\vspace{.3cm}

	\textbf{Dynamic Masking (RoBERTa):}

	\begin{itemize}
		\item Duplicate the training corpus \textit{ten} times
		\item Apply \texttt{MASK}ing procedure to each duplicate of the pre-training corpus
		\item Train for 40 epochs
		\item Model sees each training instance in ten different "versions"\\
					(each version four times) during pre-training
	\end{itemize}
	
\vfill

\end{frame}

% ------------------------------------------------------------------------------

\begin{frame}{Dynamic vs. Static Masking}

\vfill

\begin{figure}
\centering
\includegraphics[width = 8cm]{figure/roberta-dynamic.png}\\ 
\citebutton{Source: Liu et al., 2019}{https://arxiv.org/abs/1907.11692}
\end{figure}

\vfill

\end{frame}

% ------------------------------------------------------------------------------

\begin{frame}{no nsp}

\vfill

	\begin{itemize}
		\item Described as important part of the pre-training process in BERT
		\item \citebutton{Liu et al., 2019}{https://arxiv.org/abs/1907.11692} report that it hurts performance
		\item[$\to$] Especially for QNLI, MNLI, and SQuAD 1.1
		\item Conduct experiments in multiple settings: 
			\begin{itemize}
				\item SEGMENT-PAIR+NSP
				\item SENTENCE-PAIR+NSP
				\item FULL-SENTENCES
				\item DOC-SENTENCES
			\end{itemize}
	\end{itemize}
	
\vfill

\end{frame}

% ------------------------------------------------------------------------------
\begin{frame}{no nsp}

\vfill

\begin{figure}
\centering
\includegraphics[width = 11cm]{figure/roberta-nsp.png}\\ 
\citebutton{Source: Liu et al., 2019}{https://arxiv.org/abs/1907.11692}
\end{figure}

\vfill

\scriptsize
\textit{Note:} XLNet: see next Chapter.

\end{frame}

% ------------------------------------------------------------------------------

\begin{frame}{batch size}

\vfill

\begin{figure}
\centering
\includegraphics[width = 8cm]{figure/roberta-undertrained.png}\\ 
\citebutton{Source: Liu et al., 2019}{https://arxiv.org/abs/1907.11692}
\end{figure}

\vfill

\end{frame}

% ------------------------------------------------------------------------------

\begin{frame}{Changes in pre-training}

\vfill

\begin{figure}
\centering
\includegraphics[width = 11cm]{figure/roberta-undertrained2.png}\\ 
\citebutton{Source: Liu et al., 2019}{https://arxiv.org/abs/1907.11692}
\end{figure}

\vfill

\scriptsize
\textit{Note:} XLNet: see next Chapter.

\end{frame}

% ------------------------------------------------------------------------------
\begin{frame}{RoBERTa \href{https://arxiv.org/pdf/1907.11692.pdf}{\beamergotobutton{Liu et al., 2019}}}

\textbf{Architectural differences:}

\begin{itemize}
\item Architecture (layers, heads, embedding size) identical to BERT
\item 50k token BPE vocabulary instead of 30k
\item Model size differs (due to the larger embedding matrix)\\
			$\Rightarrow$ $\sim$ 125M (360M) for the BASE (LARGE) variant 
\end{itemize}

\textbf{Performance differences:}

\begin{figure}
\centering
\includegraphics[width = 11cm]{figure/roberta-sota.png}\\ 
\citebutton{Source: Liu et al., 2019}{https://arxiv.org/abs/1907.11692}
\end{figure}

\vfill

\scriptsize
\textit{Note:} Liu et al. (2019) report the accuracy for QQP while Devlin et al. (2019) report the F1 score (cf. results displayed in chapter 6.2.3); XLNet: see next Chapter.

\end{frame}

% ------------------------------------------------------------------------------

\begin{frame}{Size of embedding and hidden layer}

\vfill

	\textbf{Disentanglement of $E$ and $H$}

	\begin{itemize}
		\item	WordPiece-Embeddings (size $E$) 
			\begin{itemize}
				\item first layer of the model
				\item each token is initially mapped to this embedding
				\item context-independent
			\end{itemize}
		\item In Transformer/BERT: 
			\begin{itemize}
				\item $H = E$
				\item down-project $E$ to keys, queries and values of size $H/A$
				\item concatenate resulting embeddings from all $A$ heads
				\item results in hidden layer representation of size $H$
			\end{itemize}
		\item Implications?
	\end{itemize}

\vfill

\end{frame}

% ------------------------------------------------------------------------------

\begin{frame}{Thoughts / Implications}

\vfill

	\begin{itemize}
		\item	WordPiece-Embeddings (size $E$) 
			\begin{itemize}
				\item required representational capacity?
				\item probably could be limited w/o loosing much
			\end{itemize}
		\item Hidden-Layer-Embedding (size $H$)
			\begin{itemize}
				\item required representational capacity?
				\item depending on how polysemous a word/token might be
				\item difficult to say "one size fits all"
				\item probably might be better to rather increase this, compared to the WordPiece embeddings
			\end{itemize}
	\end{itemize}
	
\vspace{.5cm}

$\to$ \textit{Setting $E = H$ does not allow us to pursue these considerations}

\vfill

\end{frame}

% ------------------------------------------------------------------------------

\begin{frame}{Disentanglement solves this}

\vfill

	\begin{itemize}
		\item Hidden-Layer-Embeddings (size $H$) context-dependent\\
					$\to$ providing more capacity makes more sense here
		\item Setting $H >> E$ enlargens model capacity in the hidden layers without increasing the size of the embedding matrix
		\item $O(V \times H) > O(V \times E +  E \times H)$ if $H >> E$
\end{itemize}

\vfill

\end{frame}

% ------------------------------------------------------------------------------

\begin{frame}{Cross-Layer parameter sharing}

\vfill

	\begin{itemize}
		\item Typically pre-trained transformer-based models are deep and thus have many parameters
		\item Sharing them as a way to gain parameter efficiency
		\item Two different places in the network, where sharing can be done
			\begin{itemize}
				\item Attention parameters
				\item FFN parameters
				\item (or both)
			\end{itemize}
		\item Ablations: both; both individually; none
	\end{itemize}

	\begin{figure}
		\centering
		\includegraphics[width = 11cm]{figure/albert-param-sharing.png}\\ 
		\footnotesize{Source:} \href{https://arxiv.org/pdf/1909.11942.pdf}{\footnotesize Lan et al. (2019)}
	\end{figure}

\vfill

\end{frame}

% ------------------------------------------------------------------------------

\begin{frame}{Observations}

\vfill

	\begin{figure}
		\centering
		\includegraphics[width = 11cm]{figure/albert-param-sharing.png}\\ 
		\footnotesize{Source:} \href{https://arxiv.org/pdf/1909.11942.pdf}{\footnotesize Lan et al. (2019)}
	\end{figure}

	\begin{itemize}
		\item (Drastic) reduction of model size (more for sharing FFN weights)
		\item Sharing parameters hurts performance
			\begin{itemize}
				\item Worse for models with larger $E$
				\item Worse for sharing FNN compared to Attention weights
				\item[$\to$] \textbf{Why?}
			\end{itemize}
	\end{itemize}


\vfill

\end{frame}

% ------------------------------------------------------------------------------

\begin{frame}{Cross-Layer parameter sharing}

\vfill

	\begin{figure}
		\centering
		\includegraphics[width = 11cm]{figure/albert-param-sharing2.png}\\ 
		\footnotesize{Source:} \href{https://arxiv.org/pdf/1909.11942.pdf}{\footnotesize Lan et al. (2019)}
	\end{figure}

\vfill

\end{frame}

% ------------------------------------------------------------------------------

\begin{frame}{Changes in pre-training}

\vfill

	\textbf{Change/Substitution of the NSP objective}
		
	\begin{itemize}
			\item Previous works questioned the usefulness of NSP
			\item[$\to$] Lan et al. assume that this is due to lacking difficulty
			\item Introduction of \textit{Sentence-Order Prediction} (SOP) as a new pre-training task
			\item Positive examples created alike to those from NSP (take two consecutive sentences from the same document)
			\item Negative examples: Just swap the ordering of sentences
	\end{itemize}

\vfill

\end{frame}

% ------------------------------------------------------------------------------

\begin{frame}{Changes in pre-training}

\vfill

	\textbf{Illustration:}
		
	\begin{figure}
		\centering
		\includegraphics[width = 10cm]{figure/albert-sop.png}\\ 
		\footnotesize{Source:} \href{https://amitness.com/2020/02/albert-visual-summary/}{\footnotesize Amit Chaudhary}
	\end{figure}

	\textbf{Effectiveness:}
		
	\begin{figure}
		\centering
		\includegraphics[width = 11cm]{figure/albert-sop-ablation.png}\\ 
		\footnotesize{Source:} \href{https://arxiv.org/pdf/1909.11942.pdf}{\footnotesize Lan et al. (2019)}
	\end{figure}

\vfill

\end{frame}

% ------------------------------------------------------------------------------

\begin{frame}{Changes in pre-training}

\vfill

	\textbf{$n-gram$ masking for the MLM task}
		
	\begin{itemize}
			\item During pre-training BERT single tokens are masked
			\item Lan et al. mask up to three consecutive tokens
			\item Choice of $n$:\\
						\begin{center}
							\includegraphics[width = 6cm]{figure/albert-choice-n.png}
						\end{center}
	\end{itemize}

\vfill

\end{frame}

% ------------------------------------------------------------------------------

\begin{frame}{ALBERT}

	\textbf{Performance differences:}

	\begin{figure}
		\centering
		\includegraphics[width = 11cm]{figure/albert-sota.png}\\ 
		\footnotesize{Source:} \href{https://arxiv.org/pdf/1909.11942.pdf}{\footnotesize Lan et al. (2019)}
	\end{figure}

	\textbf{Notes:}

	\begin{itemize}
		\item In General: Smaller model size (because of parameter sharing)
		\item Nevertheless: Scale model up to almost similar size\\(\texttt{xxlarge} version)
		\item Strong performance compared to BERT
	\end{itemize}
\end{frame}

% ------------------------------------------------------------------------------

\endlecture
\end{document}
