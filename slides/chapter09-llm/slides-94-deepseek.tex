\documentclass[11pt,compress,t,notes=noshow, xcolor=table]{beamer}
\usepackage{xspace}
\input{../../style/preamble_new}
\input{../../latex-math/basic-math.tex}
\input{../../latex-math/basic-ml.tex}

\title{Deep Learning for NLP}

\definecolor{texblue}{rgb}{0, 0, 1}
\def\myblue#1{\textcolor{texblue}{#1}}
\long\def\devour#1{}

\def\cotpfull{chain-of-thought few-shot prompting\xspace}
\def\cotp{COT-FS-P\xspace}
\def\cotpfullcap{Chain-of-thought few-shot prompting\xspace}

\begin{document}

\titlemeta{
Large Language Models (LLMs)
}{
DeepSeek Reasoning
}{
figure/reasoning,trace
}{
\item understand DeepSeek reasoning as an example
of state-of-the-art reasoning
}


\input{newway}


\begin{frame}{Sources}

\vfill

\begin{itemize}
    \item First major public
    reasoning LLM (as opposed to proprietary closed):
    DeepSeek 
\item \citebutton{DeepSeek-AI, 2025}{https://arxiv.org/abs/2501.12948}
\item \citebutton{Nature, 2025}{https://www.nature.com/articles/s41586-025-09422-z}
\end{itemize}

\vfill

\end{frame}

\begin{vbframe}{Data for SFT: 40B tokens!}

\vfill

\begin{figure}
    \centering
    \includegraphics[width=12cm]{figure/deepseekmath.png}
\end{figure}

\vfill

\end{vbframe}


\begin{vbframe}{Reinforcement learning}

\vfill

\begin{itemize}
\item In reinforcement learning, an agent takes a sequence
    of actions towards a goal and gets feedback on the
    success of that sequence at the end.
    \item Details: Next lecture
    \item For reasoning trajectories, the feedback is:
    correct or incorrect.
\item (thanks to Emma Brunskill) \begin{tabular}{l@{\hspace{0.2cm}}ll}
& supervised learning & reinforcement learning\\\hline
goal&optimal predictor & optimal policy\\ 
learn from experience? & yes & yes\\
generalization? & yes & yes\\
feedback & rich (translation) & sparse (in/correct)\\
feedback when? & immediately & delayed\\
exploration? & no & yes
\end{tabular}

\end{itemize}

\vfill

\end{vbframe}




\begin{frame}{Synthetic data generation}

\vfill
\begin{itemize}
\item 
The User asks a
question and the Assistant solves it.
\item The Assistant first
thinks about the reasoning process ``in its mind'' and then
provides the User with the answer.
\item The reasoning process and
answer are enclosed within <think>...</think> and
<answer>...</answer> tags
\item \texttt{<think> reasoning process here </think>}
\item \texttt{<answer> answer here </answer>}
\item ``We intentionally limit our constraints to this
structural format, avoiding any content-specific biases to
ensure that we can accurately observe the natural
progression of the model during the RL process.''
\item So this is not a few-shot approach!
\end{itemize}

\vfill

\end{frame}


\begin{frame}{Inference time scaling}

\vfill

\begin{itemize}
    \item For a hard problem, it often helps to think for a
    long time to come up with the solution.
    \item Inference time scaling: teach the model to think
    for a while before giving the answer.
    \item Here, thinking means that the model produces
    potentially long output, the reasoning trace, that leads
    to the correct answer.
\end{itemize}

\vfill

\end{frame}

\begin{vbframe}{Automatic verifier/verifiable rewards: AIME example}

\vfill

\begin{figure}
    \centering
    \includegraphics[width=11.5cm]{figure/aime,example}
\end{figure}

\vfill

\end{vbframe}

\begin{vbframe}{Example of a reasoning trace}

\vfill

\begin{tabular}{lp{2cm}}
    \includegraphics[width=8cm]{figure/reasoning,trace}&
\vspace{-5cm} \ques Is this chain of thought few-shot prompting? 
\end{tabular}

\vfill

\end{vbframe}


\begin{frame}{What elements of reasoning are learned?}

\vfill

\begin{itemize}
\item dynamic strategy adaptation
    \item self-verification
    \item reflection
    \item backtracking
    \item reasoning traces are getting longer
\item exploration of alternative approches
\item diverse and sophisticated reasoning behaviors
\end{itemize}

\vfill

\end{frame}


\begin{vbframe}{Aha moment}

\vfill

\begin{figure}
    \centering
    \includegraphics[width=10cm]{figure/ahamoment}
\end{figure}

\vfill

\end{vbframe}

\begin{vbframe}{Another example for Aha moment}

\vfill

\begin{figure}
    \centering
    \includegraphics[width=10cm]{figure/raspberry,wait}
\end{figure}

\vfill

\end{vbframe}


\begin{vbframe}{The rise of reflective terms}

\vfill

\begin{tabular}{ll}
    \includegraphics[width=8cm]{figure/reasonling}
  \begin{minipage}[b]{3cm}
Frequency of
 reflective terms
(‘wait’, ‘mistake’, ‘however’,
 ‘but’, ‘retry’, ‘error’,
 ‘verify’, ‘wrong’, ‘evaluate’, ‘check’)
in model output during
 training
\end{minipage}
\end{tabular}

\vfill

\end{vbframe}


\begin{vbframe}{Accuracy AIME: Above human average}

\vfill

\begin{figure}
    \centering
    \includegraphics[width=8cm]{figure/accuracy,aime.png}
\end{figure}

\vfill

\end{vbframe}


\begin{vbframe}{R1 scaling: Without explicit length signal}

\vfill

\begin{figure}
    \centering
    \includegraphics[width=10cm]{figure/r1scaling}
\end{figure}

\vfill

\end{vbframe}



\begin{vbframe}{Unfaithfulness of LLMs to true reasoning}

\vfill

\begin{figure}
    \centering
    \includegraphics[width=10cm]{figure/sixplusnine,0.png}
\end{figure}

\vfill

\end{vbframe}

\begin{vbframe}{Unfaithfulness of LLMs to true reasoning}

\vfill

\begin{figure}
    \centering
    \includegraphics[width=9cm]{figure/sixplusnine,1.png}
\end{figure}

\vfill

\end{vbframe}



\begin{frame}{Things we have not covered}

\vfill

\begin{itemize}
\item use an eco system of several models:\\
for distillation, generation, verification
\item some ``light'' human post-editing of training data
\item rejection sampling:\\ keep only high-quality,
    high-confidence training data
\item apart from math, they also do this for code and other domains
\item major hardware optimization work
\item claim: cost of training $<$ 1 000 000 million dollar
\end{itemize}

\vfill

\end{frame}

\input{newway}

\endlecture
\end{document}



% human-defined reasoning patterns may limit model
% exploration,
% unrestricted RL training can better incentivize the
% emergence
% of new reasoning capabilities in LLMs

%    •	Cold-start SFT data: collect thousands of long CoT
%    examples via few-shot prompting, direct prompting for
%    reflected/verified answers, and cleaned outputs from
%    R1-Zero, with light human post-processing for
%    readability.  
%•	Reasoning-oriented RL with
%		  rule-based, verifiable rewards (math
%		  answers in a specified format; code
%		  checked with tests).  
%•    Rejection sampling → new SFT
%			  set (curated prompts; keep only
%			  correct traces; ~600k reasoning
%			  samples), then another RL round.
			  







