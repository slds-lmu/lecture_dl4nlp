\input{../../style/preamble}
\input{../../latex-math/basic-math.tex}
\input{../../latex-math/basic-ml.tex}

\newcommand{\titlefigure}{figure/transfer_learning_taxonomy-1.png}
\newcommand{\learninggoals}{
\item Differentiate the different flavors of transfer learning
\item Understand the challenges we might be able to overcome by using transfer learning}

\title{Transfer Learning}
% \author{}
\institute{\href{https://slds-lmu.github.io/lecture_dl4nlp/}{slds-lmu.github.io/lecture\_dl4nlp}}
\date{}

\begin{document}
\lecturechapter{Basic definitions and challenges}
\lecture{Deep Learning for NLP}

% ------------------------------------------------------------------------------

\begin{frame}{What is Transfer Learning?}

\vfill

	\textbf{Wikipedia says:} \\
    \textit{"Transfer learning is a research problem in machine learning that focuses on storing knowledge gained while solving one problem and applying it to a different but related problem."}\\
		
	\vspace{.5cm}
	
	\textbf{How it works with word2vec}
	
	\begin{itemize}
		\item Train word2vec on some "fake task" (CBOW or Skip-gram)
		\item Extract the stored knowledge (a.k.a. embedding)\\
					\textit{or:} Directly download embeddings from the web 
		\item Perform a different (supervised) task using the embeddings
	\end{itemize}
	
\vfill

\end{frame}

% ------------------------------------------------------------------------------

\begin{frame}{Taxonomy of transfer learning \href{https://ruder.io/thesis/}{\beamergotobutton{Ruder, 2019}}}
	\begin{figure}
		\centering
		\includegraphics[width = 8cm]{figure/transfer_learning_taxonomy-1.png}\\ 
		\footnotesize{Source:} \href{https://ruder.io/thesis/}{\footnotesize \it Sebastian Ruder}
	\end{figure}
\end{frame}

% ------------------------------------------------------------------------------

\begin{frame}{Taxonomy of transfer learning \href{https://ruder.io/thesis/}{\beamergotobutton{Ruder, 2019}}}

\vfill

	\textbf{Transductive Transfer learning}

	\begin{itemize}
		\item Domain adaptation:\\
					$\rightarrow$ "\textit{Transfer knowledge learned from performing task A on labeled data from domain X to performing task A in domain Y.}"\\\mbox{}
		\item Cross-lingual learning:\\
					$\rightarrow$ "\textit{Transfer knowledge learned from performing task A on labeled data from language X to performing task A in language Y.}"\\\mbox{}
		\item \textit{Important:} No labeled data in target domain/language \textit{Y}.
	\end{itemize}
	
\vfill

\end{frame}

% ------------------------------------------------------------------------------

\begin{frame}{Taxonomy of transfer learning \href{https://ruder.io/thesis/}{\beamergotobutton{Ruder, 2019}}}

\vfill

	\textbf{Inductive Transfer learning}

	\begin{itemize}
		\item Multi-task learning:\\
					$\rightarrow$ "\textit{Transfer knowledge learned from performing task A on data from domain X to performing multiple (simultaneous) tasks B, C, D, .. in domain Y.}"\\\mbox{}
		\item Sequential transfer learning:\\
					$\rightarrow$ "\textit{Transfer knowledge learned from performing task A on data from domain X to performing multiple (sequential) tasks B, C, D, .. in domain Y.}"\\\mbox{}
		\item \textit{Important:} Labeled data only for task(s) from target domain \textit{Y}.
	\end{itemize}
	
\vfill

\end{frame}

% ------------------------------------------------------------------------------

\begin{vbframe}{Challenges I}

\vfill

\textbf{Low-resource environments:}



\vfill

\end{vbframe}

% ------------------------------------------------------------------------------

\begin{vbframe}{Challenges II}

\vfill

\textbf{Cross-lingual transfer:}

\begin{itemize}
	\item Languages can be grouped into certain families
	\item Patterns that a model learns for one language, might be beneficial for learning a second language 
				(just as it is for us humans as well: For those who learned French in high school, learning Spanish
				afterwards might be easier)
	\item Again: Scarcity of resources; assume the following scenario:
		\begin{itemize}
			\item \textbf{Large} parallel corpus for languages A and B
			\item \textbf{Large} parallel corpus for languages A and C
			\item \textit{Small} parallel corpus for languages B and C
			\item[$\to$] Training a model for B and C in isolation not the best idea
		\end{itemize}
\end{itemize}

\vfill

\end{vbframe}

\endlecture
\end{document}
