\documentclass[11pt,compress,t,notes=noshow, xcolor=table]{beamer}
\input{../../style/preamble_new}
\input{../../latex-math/basic-math.tex}
\input{../../latex-math/basic-ml.tex}

\title{Deep Learning for NLP}

\begin{document}

\titlemeta{
BERT
}{
ARLMs vs. MLM
}{
figure/arlm-mlm
}{
\item Understand the concept of self-supervision
\item Gain ability to distinguish different types of language models
}

% ------------------------------------------------------------------------------

\begin{frame}{again: what is a language model?}

\vfill

\begin{itemize}
	\item Statistical model that predicts text that fits well for a given context (typically also text)
	\item Auto-regressive LMs (ARLMs)
	\begin{itemize}
		\item Predict one word that is highly likely given a prompt (previous words)
		\item For predicting an entire text, repeat the process (i.e., extend the prompt with previously predicted words)
		\item To predict a text from scratch, use an extra symbol \texttt{<START>} as the initial prompt
	\end{itemize}
\end{itemize}

\vfill

\end{frame}

% ------------------------------------------------------------------------------

\begin{frame}{arlm: toy example}
	\vfill
	\begin{figure}
		\centering
		\includegraphics[width=12cm,page=1]{figure/arlm.pdf}
	\end{figure}
	\vfill
\end{frame}
\begin{frame}[noframenumbering]{arlm: toy example}
	\vfill
	\begin{figure}
		\centering
		\includegraphics[width=12cm,page=2]{figure/arlm.pdf}
	\end{figure}
	\vfill
\end{frame}
\begin{frame}[noframenumbering]{arlm: toy example}
	\vfill
	\begin{figure}
		\centering
		\includegraphics[width=12cm,page=3]{figure/arlm.pdf}
	\end{figure}
	\vfill
\end{frame}
\begin{frame}[noframenumbering]{arlm: toy example}
	\vfill
	\begin{figure}
		\centering
		\includegraphics[width=12cm,page=4]{figure/arlm.pdf}
	\end{figure}
	\vfill
\end{frame}
\begin{frame}[noframenumbering]{arlm: toy example}
	\vfill
	\begin{figure}
		\centering
		\includegraphics[width=12cm,page=5]{figure/arlm.pdf}
	\end{figure}
	\vfill
\end{frame}
\begin{frame}[noframenumbering]{arlm: toy example}
	\vfill
	\begin{figure}
		\centering
		\includegraphics[width=12cm,page=6]{figure/arlm.pdf}
	\end{figure}
	\vfill
\end{frame}
\begin{frame}[noframenumbering]{arlm: toy example}
	\vfill
	\begin{figure}
		\centering
		\includegraphics[width=12cm,page=7]{figure/arlm.pdf}
	\end{figure}
	\vfill
\end{frame}
\begin{frame}[noframenumbering]{arlm: toy example}
	\vfill
	\begin{figure}
		\centering
		\includegraphics[width=12cm,page=8]{figure/arlm.pdf}
	\end{figure}
	\vfill
\end{frame}
\begin{frame}[noframenumbering]{arlm: toy example}
	\vfill
	\begin{figure}
		\centering
		\includegraphics[width=12cm,page=9]{figure/arlm.pdf}
	\end{figure}
	\vfill
\end{frame}
\begin{frame}[noframenumbering]{arlm: toy example}
	\vfill
	\begin{figure}
		\centering
		\includegraphics[width=12cm,page=10]{figure/arlm.pdf}
	\end{figure}
	\vfill
\end{frame}
\begin{frame}[noframenumbering]{arlm: toy example}
	\vfill
	\begin{figure}
		\centering
		\includegraphics[width=12cm,page=11]{figure/arlm.pdf}
	\end{figure}
	\vfill
\end{frame}
\begin{frame}[noframenumbering]{arlm: toy example}
	\vfill
	\begin{figure}
		\centering
		\includegraphics[width=12cm,page=12]{figure/arlm.pdf}
	\end{figure}
	\vfill
\end{frame}
\begin{frame}[noframenumbering]{arlm: toy example}
	\vfill
	
	\begin{figure}
		\centering
		\includegraphics[width=12cm,page=13]{figure/arlm.pdf}
	\end{figure}
	
	\vfill
\end{frame}

% ------------------------------------------------------------------------------

\begin{vbframe}{arlm: probabilistic interpretation}

\vfill

\begin{itemize}
	\item Gives an estimate for the probability of a sentence using conditional probabilities
	\item In general: $$ P(A \cap B) = P(B) \cdot P(A | B)$$
	\item For $P(sentence)$:
	$$ P(w_1, w_2, \dots, w_n | w_0)$$ $$ = P(w_1 | w_0) \cdot P(w_2 | w_0, w_1) \cdot \hdots \cdot P(w_n | w_0, \dots, w_{n-1})$$
\end{itemize}

\vfill

\end{vbframe}


% ------------------------------------------------------------------------------

\begin{vbframe}{Examples of ARLMs (1)}

\vfill

\textbf{Neural Machine Translation:}

	\begin{figure}
		\centering
		\includegraphics[width=12cm]{figure/arlm1.png}
	\end{figure}
	
\vfill

\end{vbframe}

% ------------------------------------------------------------------------------

\begin{vbframe}{Examples of ARLMs (1)}

\vfill

\textbf{Image Captioning:}

	\begin{figure}
		\centering
		\includegraphics[width=8cm]{figure/arlm2.png}
	\end{figure}
\vfill

\end{vbframe}

% ------------------------------------------------------------------------------

\begin{vbframe}{Masked language models (MLM)}

\vfill

\begin{itemize}
	\item We have seen auto-regressive LMs
			\begin{itemize}
				\item context: previous words
				\item predict: next word
			\end{itemize}
	\item Another type: \textit{Masked} LMs (MLMs)
			\begin{itemize}
				\item context: surrounding words
				\item predict: masked word
			\end{itemize}
\end{itemize}

\vfill

\end{vbframe}

% ------------------------------------------------------------------------------

\begin{vbframe}{mlm: toy example}

\vfill

	\begin{figure}
		\centering
		\includegraphics[width=10cm]{figure/mlm.pdf}
	\end{figure}
	
\vfill

\end{vbframe}

% ------------------------------------------------------------------------------

\begin{vbframe}{mlm: probabilistic interpretation}

\vfill

\begin{itemize}
	\item Estimates $P(w_i | w_0, w_1, w_{i-1}, \dots, w_{i+1}, w_n)$
	\item No ``clean`` estimate for $P(sentence)$, as
	$$ P(w_1, w_2, \dots, w_n | w_0)$$ $$\neq P(w_1 | w_0, w_2, \dots, w_n) \cdot P(w_2 | w_0, w_1, w_3, \dots, w_n)$$
	\item ARLMs are better than MLMs for generating texts 
	\item Advantage of MLMs: Learning contextualized representations
\end{itemize}

\vfill

\end{vbframe}

% ------------------------------------------------------------------------------

\begin{vbframe}{self-supervision}

\vfill

	\begin{figure}
		\centering
		\includegraphics[width=10cm]{figure/selfsup.jpeg}
	\end{figure}
	
\vfill

\end{vbframe}

% ------------------------------------------------------------------------------

\begin{frame}{Definition}

	\vfill 
	
	\textit{Unsupervised Learning:}

	\begin{itemize}
		\item No labels attached to the data
		\item Learn patterns / clusters from the features only
	\end{itemize}
	
	\textit{Supervised Learning:}

	\begin{itemize}
		\item (Gold) Labels attached to the data
		\item Learn from the association between features and labels
	\end{itemize}
	
	\textbf{Self-Supervised Learning:}

	\begin{itemize}
		\item No \textit{external} labels attached to the data\\
					$\to$ Samples with suitable labels can be generated from the known structure of the data itself 
		\item \textit{Technically} supervised learning, but \textit{no labeling effort} $+$ simultan-\\eous ability to generate massive amounts of labeled data points
	\end{itemize}
	
	\vfill 
	
\end{frame}

% ------------------------------------------------------------------------------

\begin{vbframe}{Self-supervised objectives}

\vfill

\textbf{Self-supervised objectives:}
	
	\begin{itemize}
		\item Skip-gram objective (cf. word2vec)
		\item Language modeling objective
		\item \textit{Masked language modeling (MLM)} objective
		\item ... and many more possibilities for text data
	\end{itemize}
\vfill

\end{vbframe}

% ------------------------------------------------------------------------------

\endlecture
\end{document}