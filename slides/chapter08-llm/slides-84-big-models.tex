\input{../../style/preamble}
\input{../../latex-math/basic-math.tex}
\input{../../latex-math/basic-ml.tex}
\newcommand*\POS[1]{\textsubscript{\texttt{#1}}} % tag with part of speech
\usepackage{qtree} %parse tree

\newcommand{\titlefigure}{figure/tasks.png}
\newcommand{\learninggoals}{
\item defined the key learning goals here
\item second learning goal}

\title{chapter title}
% \author{}
\institute{\href{https://slds-lmu.github.io/lecture_dl4nlp/}{slds-lmu.github.io/lecture\_dl4nlp}}
\date{}

\begin{document}
\lecturechapter{specific lecture title}
\lecture{Deep Learning for NLP} % stays constant

% ------------------------------------------------------------------------------

\begin{vbframe}{Categorization of NLP Tasks}

\vfill

\textbf{"Low-Level" tasks:}

	\begin{itemize}
		\item some text
		\item would be nice to use these visual separations between slides (see below)\\
					\%--------------
	\end{itemize}

\vfill

\end{vbframe}

% ------------------------------------------------------------------------------

\begin{vbframe}{Examples of low-level NLP tasks}

\textbf{Sequence tagging}

\begin{itemize}
	\item POS-tagging (part of speech)
\end{itemize}
	
\begin{exampleblock}{Example}
	Time flies   like   an   arrow.\\Fruit   flies   like   a   banana.
\end{exampleblock}

\vfill

\end{vbframe}

% ------------------------------------------------------------------------------

\begin{vbframe}{Examples of low-level NLP tasks}


\textbf{Sequence tagging}

\begin{itemize}
	\item POS-tagging (part of speech)
\end{itemize}

\begin{exampleblock}{Example}
		Time\POS{NN} flies\POS{VBZ}  like\POS{IN}   an\POS{DT}   arrow\POS{NN}.		\\ Fruit\POS{NN}   flies\POS{NN}   like\POS{VB}   a\POS{DT}   banana\POS{NN}.
\end{exampleblock}

\begin{footnotesize}
IN = Preposition or subordinating conjunction (conjunction here); VBZ = Verb, 3rd person singular present; DT = determiner; NN = singular noun
\end{footnotesize}

\vfill

\end{vbframe}

% ------------------------------------------------------------------------------

\begin{vbframe}{a slide with a figure}

\vfill

	\begin{figure}
		\centering
		\includegraphics[width = 8cm]{figure/toklevel.png}\\ 
		\footnotesize{Source:} \href{https://arxiv.org/pdf/1810.04805.pdf}{\footnotesize \it Devlin et al. (2018)}
	\end{figure}

\vfill

\end{vbframe}

% ------------------------------------------------------------------------------
