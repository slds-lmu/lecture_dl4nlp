\input{../../style/preamble}
\input{../../latex-math/basic-math.tex}
\input{../../latex-math/basic-ml.tex}

\newcommand{\titlefigure}{figure/babel.png}
\newcommand{\learninggoals}{
\item Understand why we need multilingual models
\item Understand the challenges
\item Brainstorm solutions
}

\title{Multilinguality}
% \author{}
\institute{\href{https://slds-lmu.github.io/lecture_dl4nlp/}{slds-lmu.github.io/lecture\_dl4nlp}}
\date{}

\begin{document}
\lecturechapter{}
\lecture{Multilinguality}



\begin{vbframe}{Acknowledgements}

\vfill

\begin{itemize}
\item This presentation is based on slides originally authored by:
  \begin{itemize}
\item Hinrich Sch\"{u}tze
\end{itemize}

\item \url{https://slds-lmu.github.io/dl4nlp/}
\end{itemize}

\vfill

\end{vbframe}


% ------------------------------------------------------------------------------






\begin{vbframe}{Multilinguality lecture}

\vfill

\textbf{Roadmap}

	\begin{itemize}
		\item Motivation \& challenges
		\item Discussion

	\end{itemize}

\vfill

\end{vbframe}



\section{Motivation: Bali slides 1--30}

\section{Discussion}

\begin{vbframe}{What would you do?}

\vfill

	\begin{itemize}
\item Given: the Glot500 dataset
	\begin{itemize}
        \item Data for 500 languages, a lot for some, only
        30 000 sentences for others
\end{itemize}
\item Goal: Create (an) autoregressive language model(s) for
        all 500 languages
\item Question: What is the best approach?
\item Work in groups
\end{itemize}

\vfill

\end{vbframe}


\begin{vbframe}{OpenAI's take on LLM optimization}

\vfill

\textbf{Can this help?}

\textbf{\href{https://www.youtube.com/watch?v=ahnGLM-RC1Y}{\beamergotobutton{OpenAI's
                current take}}}


\vfill

\begin{figure}
\centering
\includegraphics[width = 9cm]{figure/openai,llm,optimization.png}
\end{figure}

\end{vbframe}


\endlecture
\end{document}

