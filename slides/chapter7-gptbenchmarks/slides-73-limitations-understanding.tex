\input{../../style/preamble}
\input{../../latex-math/basic-math.tex}
\input{../../latex-math/basic-ml.tex}

%\newcommand{\titlefigure}{figure/gpt_sq.png}
\newcommand{\learninggoals}{
\item Recap GPT and the ideas behind standard language modelling
\item Understand the difference between fine-tuning and X-shot learning}
\definecolor{texblue}{rgb}{0, 0, 1}
\def\myblue#1{\textcolor{texblue}{#1}}

\title{Intro to GPT \& X-shot learning}
% \author{}
\institute{\href{https://slds-lmu.github.io/lecture_dl4nlp/}{slds-lmu.github.io/lecture\_dl4nlp}}
\date{}

\begin{document}
\lecturechapter{GPT \&  Benchmarks}
\lecture{Deep Learning for NLP}

% ------------------------------------------------------------------------------

\begin{vbframe}{GPT}

\vfill

  \begin{itemize}
\item Like BERT, GPT is a language model.
\item But not MLM, but a conventional language model: it predicts
the next word (or subword).
\item Like BERT, GPT is trained on a huge corpus,
actually an even huger corpus.
\item Like BERT, GPT is a transformer architecture.
\item Difference 1: GPT is a \myblue{single model}
that aims to solve \myblue{all tasks}.  
  \begin{itemize}
    \item It can also switch back and forth between tasks and solve
    tasks within tasks, another human capability that is
    important in practice. \myblue{``fluidity''}
    \end{itemize}
\item Difference 2: GPT leverages \myblue{task descriptions}.
\item Difference 3: GPT is effective at
\myblue{few-shot learning}.
    \end{itemize}

\vfill

\end{vbframe}

% ------------------------------------------------------------------------------

\begin{vbframe}{GPT: Two types of learning}

\vfill

\begin{figure}
		\centering
		\includegraphics[width=11cm]{figure/twotypesoflearning.png}
	\end{figure}

\vfill

\end{vbframe}

% ------------------------------------------------------------------------------

\begin{vbframe}{GPT: Effective in-context learning}

\vfill
			
	\begin{figure}
		\centering
		\includegraphics[width=11cm]{figure/incontextlearning.png}
	\end{figure}
\vfill

\end{vbframe}

% ------------------------------------------------------------------------------

\begin{vbframe}{X-shot comparison and effect of larger corpora}

\vfill

	\begin{figure}
		\centering
		\includegraphics[width=11cm]{figure/xshotlargecorpora.png}
	\end{figure}

\vfill

\end{vbframe}

% ------------------------------------------------------------------------------

\begin{vbframe}{Fine-tuning (not used by GPT)}

\vfill

	\begin{figure}
		\centering
		\includegraphics[height=7cm,width=6cm]{figure/gptnofinetuning.png}
	\end{figure}

\vfill

\end{vbframe}

% ------------------------------------------------------------------------------

\begin{vbframe}{Zero-shot (no gradient update)}

\vfill

	\begin{figure}
		\centering
		\includegraphics[width=11cm]{figure/gptzeroshot.png}
	\end{figure}

\vfill

\end{vbframe}


% ------------------------------------------------------------------------------

\begin{vbframe}{One-shot (no gradient update)}

\vfill

	\begin{figure}
		\centering
		\includegraphics[width=11cm]{figure/gptoneshot.png}
	\end{figure}

\vfill

\end{vbframe}


% ------------------------------------------------------------------------------

\begin{vbframe}{Few-shot (no gradient update)}

\vfill

	\begin{figure}
		\centering
		\includegraphics[width=11cm]{figure/gptfewshot.png}
	\end{figure}

\vfill

\end{vbframe}


% ------------------------------------------------------------------------------

\begin{vbframe}{Architecture}

\vfill

	\begin{figure}
		\centering
		\includegraphics[width=11cm]{figure/gptarch.png}
	\end{figure}

\vfill

\end{vbframe}


% ------------------------------------------------------------------------------

\begin{vbframe}{Training corpus}

\vfill

	\begin{figure}
		\centering
		\includegraphics[width=11cm]{figure/traincorp.png}
	\end{figure}
\vfill

\end{vbframe}



% ------------------------------------------------------------------------------

\begin{vbframe}{Loss as a function of compute}

\vfill

	\begin{figure}
		\centering
		\includegraphics[width=9cm]{figure/losscompute.png}
	\end{figure}

\vfill

\end{vbframe}


\endlecture
\end{document}
